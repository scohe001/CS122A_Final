%%%%%%%%%%%%%%%%%%%%%%%%%%%%%%%%%%%%%%%%%
% University Assignment Title Page 
% LaTeX Template
% Version 1.0 (27/12/12)
%
% This template has been downloaded from:
% http://www.LaTeXTemplates.com
%
% Original author:
% WikiBooks (http://en.wikibooks.org/wiki/LaTeX/Title_Creation)
%
% License:
% CC BY-NC-SA 3.0 (http://creativecommons.org/licenses/by-nc-sa/3.0/)
% 
% Instructions for using this template:
% This title page is capable of being compiled as is. This is not useful for 
% including it in another document. To do this, you have two options: 
%
% 1) Copy/paste everything between \begin{document} and \end{document} 
% starting at \begin{titlepage} and paste this into another LaTeX file where you 
% want your title page.
% OR
% 2) Remove everything outside the \begin{titlepage} and \end{titlepage} and 
% move this file to the same directory as the LaTeX file you wish to add it to. 
% Then add \input{./title_page_1.tex} to your LaTeX file where you want your
% title page.
%
%%%%%%%%%%%%%%%%%%%%%%%%%%%%%%%%%%%%%%%%%
%\title{Title page with logo}
%----------------------------------------------------------------------------------------
%   PACKAGES AND OTHER DOCUMENT CONFIGURATIONS
%----------------------------------------------------------------------------------------

\documentclass[12pt]{article}
\usepackage[english]{babel}
\usepackage[utf8x]{inputenc}
\usepackage{amsmath}
\usepackage{graphicx}
\usepackage[colorinlistoftodos]{todonotes}
\usepackage{hyperref}
\usepackage{titlesec}

\begin{document}

\begin{titlepage}

\newcommand{\HRule}{\rule{\linewidth}{0.5mm}} % Defines a new command for the horizontal lines, change thickness here

\center % Center everything on the page
 
%----------------------------------------------------------------------------------------
%   HEADING SECTIONS
%----------------------------------------------------------------------------------------

\textsc{\LARGE University California Riverside}\\[1.5cm] % Name of your university/college
\textsc{\Large CS122A Final Project}\\[0.5cm] % Major heading such as course name
\textsc{\large Fall 2016}\\[0.5cm] % Minor heading such as course title

%----------------------------------------------------------------------------------------
%   TITLE SECTION
%----------------------------------------------------------------------------------------

\HRule \\[0.4cm]
{ \huge \bfseries Cube Solver}\\[0.4cm] % Title of your document
\HRule \\[1.5cm]
 
%----------------------------------------------------------------------------------------
%   AUTHOR SECTION
%----------------------------------------------------------------------------------------

\begin{minipage}{0.4\textwidth}
\begin{flushleft} \large
\emph{Author:}\\
Stanley \textsc{Cohen} % Your name
\end{flushleft}
\end{minipage}
~
\begin{minipage}{0.4\textwidth}
\begin{flushright} \large
\emph{Professor:} \\
Jeffery \textsc{McDaniels} % Supervisor's Name
\end{flushright}
\end{minipage}\\[2cm]

% If you don't want a supervisor, uncomment the two lines below and remove the section above
%\Large \emph{Author:}\\
%John \textsc{Smith}\\[3cm] % Your name

%----------------------------------------------------------------------------------------
%   DATE SECTION
%----------------------------------------------------------------------------------------

{\large \today}\\[2cm] % Date, change the \today to a set date if you want to be precise

%----------------------------------------------------------------------------------------
%   LOGO SECTION
%----------------------------------------------------------------------------------------
\includegraphics[scale=.5]{logo.png}\\[1cm] % Include a department/university logo - this will require the graphicx package
%----------------------------------------------------------------------------------------

\vfill % Fill the rest of the page with whitespace

\end{titlepage}


\begin{abstract}
The Cube Solver is a Rubik’s Cube solver that uses a Raspberry Pi and OpenCV for reading in a cube. The solution is then found using Kociemba’s algorithm and passed over to an Atmega1284 to solve the cube for the lazy layman.
\end{abstract}

\section{User Guide}

Turn on the Atmega1284 and ensure that the SPI lines are connected to the Raspberry Pi. Launch the $cube\_reader.py$ Python script on the Pi. Show the camera the sides of the cube listed in the terminal and press enter to lock in a side. Once all sides have been entered, type “done.” Place the cube into the arms of the solver and press the button on the breadboard to begin running the solution! 

\section{Technologies and Components}

\begin{itemize}
	\item AVR Studio 6
	\item Atmega1284
	\item 8 Car Grade Solenoids
	\item 4 Stepper motors
	\item 8 Relays (to form H-Bridges for the solenoids)
	\item Raspberry Pi
	\item Pi Camera
	\item OpenCV
\end{itemize}

\section{Demo Video}

Demo video can be found \href{https://youtu.be/lqFNS5mc-u4}{here}.

\section{Source Code}
\label{sec:code}

\subsection{Raspberry Pi}

\subsubsection{cube\_reader.py}

This script is used to read the cube via the Pi Cam with the help of OpenCV and then convert the 6x3x3 array of cubies into a cube string to be passed to Kociemba’s algorithm. It will then convert the returned solution into a set of moves that can be executed on my machine (only L, R, F, B moves) and pass the moves over SPI to the Atmega.

\subsubsection{myspi.cpp}

This C++ code will send whatever value it's passed over the Raspberry Pi's SPI. The majority of the code is from MontaVista Software. I modified their template to fit my purpose.

\subsection{Atmega1284}

\subsubsection{queue.h}

Used to store the set of moves we need to execute. Allows us to receive multiple moves and hold them until a button is pressed so the user can have time to place the cube into the machine.

\subsubsection{joystick.h}

Used for debugging to put moves into the queue so the Raspberry Pi does not need to be on and hooked up at all times to test everything. 

\subsubsection{main.c}

Listen for moves sent over SPI and add them to the move queue. Control the 4 stepper motors over 2 shift registers. Control the 8 solenoids to pull back the correct arm after executing a move. Execute all moves in the move queue on button press.

\section{Wiring}

\begin{center}
  \begin{tabular}{ | l | c | c | }
    \hline
    Wire & Atmega & Pi \\ \hline
    SPI MOSI & PB5 & PIN19 \\ 
    SPI MISO & PB6 & PIN21 \\ 
    SPI CLK & PB7 & PIN23  \\ 
    GND & GND & PIN25 \\ \hline
    SHIFT1 RCLK & PC1 & - \\
    SHIFT1 SRCLK & PC2 & - \\
    SHIFT1 SRCLR & PC3 & - \\
    SHIFT1 SER & PC0 & - \\ \hline
    SHIFT2 RCLK & PC1 & - \\
    SHIFT2 SRCLK & PC2 & - \\
    SHIFT2 SRCLR & PC3 & - \\
    SHIFT2 SER & PC4 & - \\ \hline
    MOTOR1 & SHIFT1[3:0] & - \\ 
    MOTOR2 & SHIFT1[7:4] & - \\ 
    MOTOR3 & SHIFT2[3:0] & - \\
    MOTOR4 & SHIFT2[7:4] & - \\ \hline
    SOLENOID1 H-BRIDGE & PD[1:0] & - \\
    SOLENOID2 H-BRIDGE & PD[3:2] & - \\
    SOLENOID3 H-BRIDGE & PD[5:4] & - \\
    SOLENOID4 H-BRIDGE & PD[7:6] & - \\ \hline
    BUTTON & PA5 & - \\ \hline
    JOYSTICK & PA[1:0] & - \\ \hline

    \hline
  \end{tabular}
\end{center}

\end{document}